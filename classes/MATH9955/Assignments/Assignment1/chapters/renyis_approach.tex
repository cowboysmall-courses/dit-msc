
\chapter{R\'enyi's Approach}











\section{Solution of the Delay Differential Equation}

A more satisfying approach is R\'enyi's solution to the delay differential equation derived 
from the master equation (see \cite{yoshiaki2011random}). We start with the familiar form: \bigskip

\begin{eqnarray} \label{eq:0}
	M(x + 1) = 1 + \frac{2}{x} \int_{0}^{x} M(t) dt
\end{eqnarray}\medskip

then multiply both sides by $x$: \bigskip

\[
	x M(x + 1) = x + 2 \int_{0}^{x} M(t) dt
\]\medskip

and differentiate the result with respect to x: \bigskip

\begin{eqnarray} \label{eq:7}
	x M^{\prime}(x + 1) + M(x + 1) = 1 + 2 M(x) \quad \text{for} \quad x > 0
\end{eqnarray}\medskip

We now consider the following: \bigskip

\begin{eqnarray} \label{eq:8}
	\varphi(s) = \int_{0}^{\infty} M(x) e^{-sx} dx
\end{eqnarray}\medskip

which is the Laplace transform of $M(x)$. We will show that $\varphi(s)$ takes the form: \bigskip

\begin{eqnarray} \label{eq:9}
	\varphi(s) = \frac{e^{-s}}{s^2} \int_{s}^{\infty} \exp \left( -2 \int_{s}^{t} \frac{1 - e^{-u}}{u} du \right) dt
\end{eqnarray}\medskip

Returning to equation \ref{eq:7} with the initial condition: \bigskip

\[
	M(x) = 0 \quad \text{for} \quad 0 \leq x < 1
\]\medskip

we multiply across by $e^{-sx}$ and integrating: \bigskip

\begin{eqnarray*}
	\int_{0}^{\infty} x M^{\prime}(x + 1) e^{-sx} dx + \int_{0}^{\infty} M(x + 1) e^{-sx} dx & = & \int_{0}^{\infty} e^{-sx} dx + \int_{0}^{\infty} 2 M(x) e^{-sx} dx \\\\
																							 & = & \left. -\frac{ e^{-sx}}{s} \right|_{x = 0}^{\infty} + 2 \int_{0}^{\infty} M(x) e^{-sx} dx \\\\
																							 & = & \frac{1}{s} + 2 \varphi(s) 
\end{eqnarray*}\medskip

We take each part of the left hand side separately: \bigskip

\begin{eqnarray*}
	\int_{0}^{\infty} M(x + 1) e^{-sx} dx & = & \int_{1}^{\infty} M(t) e^{-s(t - 1)} dt \\\\
										  & = & e^{s} \int_{1}^{\infty} M(t) e^{-st} dt \\\\
										  & = & e^{s} \varphi(s) 
\end{eqnarray*}\medskip

making use of our initial condition. Looking at the other integral, and integrating by parts: \bigskip

\begin{eqnarray*}
	\int_{0}^{\infty} x M^{\prime}(x + 1) e^{-sx} dx & = & - \frac{d}{ds} \left( \int_{0}^{\infty} M^{\prime}(x + 1) e^{-sx} dx \right) \\\\
													 & = & - \frac{d}{ds} \left( \left. M(x + 1) e^{-sx} \right|_{x = 0}^{\infty} + s \int_{0}^{\infty} M(x + 1) e^{-sx} dx \right) \\\\
													 & = & - \frac{d}{ds} \left( - M(1) + s e^{s} \varphi(s) \right) \\\\
													 & = & - \frac{d}{ds} \left( s e^{s} \varphi(s) \right) 
\end{eqnarray*}\medskip

so our delay differential equation becomes: \bigskip

\begin{eqnarray*}
	- \frac{d}{ds} \left( s e^{s} \varphi(s) \right) + e^{s} \varphi(s) & = & \frac{1}{s} + 2 \varphi(s) \\\\
					   - \frac{d}{ds} \left( s e^{s} \varphi(s) \right) & = & \frac{1}{s} + 2 \varphi(s) - e^{s} \varphi(s) \\\\
						 \frac{d}{ds} \left( s e^{s} \varphi(s) \right) & = & \varphi(s) ( e^{s} - 2 ) - \frac{1}{s}
\end{eqnarray*}\medskip

we now make the substitution $w(s) = e^{s} \varphi(s)$. Our equation becomes: \bigskip

\begin{eqnarray*}
	\frac{d}{ds} \left( s w(s) \right) & = & w(s) ( 1 - 2 e^{-s}) - \frac{1}{s} \\\\
				w(s) + s w^{\prime}(s) & = & w(s) - 2 w(s) e^{-s} - \frac{1}{s} \\\\
					   s w^{\prime}(s) & = & - 2 w(s) e^{-s} - \frac{1}{s} 
\end{eqnarray*}\medskip

%now we once again make use of the fact that $0 \leq M(x) \leq x$ and find that: \bigskip
%
%\begin{eqnarray*}
%	\varphi(s) &    = & \int_{0}^{\infty} M(x) e^{-sx} dx \\\\
%			   & \leq & \int_{0}^{\infty} x e^{-sx} dx \\\\
%			   & \leq & e^{-s} \left( \frac{1}{s} + \frac{1}{s^2} \right)
%\end{eqnarray*}\medskip

solving this inhomogeneous equation gives us: \bigskip

\[
	w(s) = \frac{1}{s^2} \int_{s}^{\infty} \exp \left( -2 \int_{s}^{t} \frac{1 - e^{-u}}{u} du \right) dt
\]\medskip

which is another form of equation \ref{eq:9}.










\section{The Computation of the Limit}

R\'enyi provides the following theorem: \bigskip 

\begin{theorem}
	One has the limit
	\[
		\lim_{x \to \infty} \frac{M(x)}{x} = C_R
	\]
	with
	\[
		C_R = \int_{0}^{\infty} \exp \left( -2 \int_{0}^{t} \frac{1 - e^{-u}}{u} du \right) dt
	\]
\end{theorem}\bigskip

In order to compute the limit, R\'enyi makes use of the following 
\emph{Tauberian theorem} (stated without proof):\bigskip

\begin{mdframed}
	\begin{theorem}
		if $f$ is positive and integrable over every finite interval $(0, T)$ and $e^{-st}f(t)$ is integrable 
		over $(0, \infty)$ for any $s > 0$ and if \bigskip
		\[
			g(s) = \int_{0}^{\infty} e^{-st} f(t) dt \simeq H s^{-\beta} \text{ as } s \to 0
		\]\medskip
		where $\beta > 0$ and $H > 0$ then when $x \to \infty$, we have \bigskip
		\[
			F(x) = \int_{0}^{x} f(t) dt \simeq \frac{H}{\Gamma(1 + \beta)}  x^{\beta} \text{ as } x \to \infty
		\]\medskip
	\end{theorem}
\end{mdframed} \bigskip


\begin{proof}
	If we re-arrange equation \ref{eq:9} we have: \bigskip
	
	\[
		s^2 \varphi(s) = e^{-s} \int_{s}^{\infty} \exp \left( -2 \int_{s}^{t} \frac{1 - e^{-u}}{u} du \right) dt
	\]\medskip
	
	taking the limit as $s \to 0$: \bigskip
	
	\begin{eqnarray*}
		\lim_{s \to 0} s^2 \varphi(s) & = & \int_{0}^{\infty} \exp \left( -2 \int_{0}^{t} \frac{1 - e^{-u}}{u} du \right) dt \\\\
									  & = & C_R
	\end{eqnarray*}\medskip
	
	from the definition of $\varphi(s)$ (equation \ref{eq:8}): \bigskip
	
	\[
		\int_{0}^{\infty} M(x) e^{-sx} dx = \frac{C_R}{s^2} 
	\]\medskip
	
	as $s \to 0$. Applying the above theorem, we get: \bigskip
	
	\begin{eqnarray*}
					  \int_{0}^{x} M(t) dt & = & \frac{C_R x^2}{\Gamma(3)} \\\\
										   & = & \frac{C_R x^2}{2} \\\\
		\frac{2}{x^2} \int_{0}^{x} M(t) dt & = & C_R 
	\end{eqnarray*}\medskip
	
	as $x \to \infty$. Returning to the familiar equation \ref{eq:0} and dividing across by $x$, 
	and taking the limit as $x \to \infty$, we have: \bigskip
	
	\begin{eqnarray*}
		\lim_{x \to \infty} \frac{M(x)}{x} & = & \lim_{x \to \infty} \left( \frac{1}{x} + \frac{2}{x (x - 1)} \int_{0}^{x - 1} M(t) dt \right) \\\\
										   & = & C_R
	\end{eqnarray*}\medskip

	which is the required limit.
\end{proof}\bigskip











\section{Remarks}

While certainly more interesting than Weiner's approach - indeed an expression for the 
jamming limit is derived - this author is still left feeling dissatisfied by the approach 
taken. The Laplace transform of the master equation is introduced without any of it's 
specific features being used - for example, the reduction of an analytic problem to an 
algebraic problem, and no inverse step is performed - it is merely used as a convenient 
form in order to apply a \emph{Tauberian theorem}, and hence the reference to the Laplace 
Transform could have been omitted. \bigskip











