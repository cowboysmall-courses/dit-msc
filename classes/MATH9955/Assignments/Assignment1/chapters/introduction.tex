
\chapter{Introduction}











\section{Problem Statement}

Consider an interval $(0, x)$ upon which we place a segment of unit length 
at random. We continue by placing a second segment of unit length randomly 
upon the original interval, discarding the segment if it overlaps with the 
original one. \bigskip

We continue in this fashion until we can no longer add unit segments without 
overlap. At each step the next position within the interval is chosen from a 
uniform distribution of the remaining locations within the interval. \bigskip

We are interested in both the expected value of the number of unit segments 
contained within the interval $(0, x)$, denoted $M(x)$, and the expected 
filling density of unit segments within the interval, denoted $M(x) / x$. \bigskip






\section{Derivation of the Master Equation}

We first look at the derivation of the master equation for $M(x)$ which forms the 
heart of the problem, and will be crop up in the rest of the paper. \bigskip

We initially consider an interval $(0, x + 1)$ and upon this place a unit segment 
$(t, t + 1)$. This unit segment partitions the original interval into two smaller 
intervals - $(0, t)$ and $(t + 1, x + 1)$. The expected number of unit segments 
contained within the original interval is: \bigskip

\[
	M(x + 1) = M(t) + 1 + M(x - t)
\]\medskip

where $1$ represents the expectation of the added segment within the unit interval 
$(t, t + 1)$ - in other words, we have one unit segment within the unit interval 
$(t, t + 1)$. Integrating with respect to $t$ we get: \bigskip

\begin{eqnarray*}
	\int_{0}^{x} M(x + 1) dt & = & \int_{0}^{x} [M(t) + 1 + M(x - t)] dt \\\\
	M(x + 1) \int_{0}^{x} dt & = & \int_{0}^{x} dt + \int_{0}^{x} [M(t) + M(x - t)] dt \\\\
			M(x + 1) \cdot x & = & x + \int_{0}^{x} [M(t) + M(x - t)] dt \\\\
					M(x + 1) & = & 1 + \frac{1}{x} \int_{0}^{x} [M(t) + M(x - t)] dt 
\end{eqnarray*}\medskip

as the distributions within each of the smaller intervals are uniform, and hence 
the same, we get: \bigskip

\begin{eqnarray} \label{eq:1}
	M(x + 1) = 1 + \frac{2}{x} \int_{0}^{x} M(t) dt
\end{eqnarray}\medskip

changing variables we get: \bigskip

\begin{eqnarray} \label{eq:2}
	M(x) = 1 + \frac{2}{x - 1} \int_{0}^{x - 1} M(t) dt
\end{eqnarray}\medskip

more completely, and because adding a unit segment to an interval of length less 
than $1$ has no meaning, the equation for $M(x)$ can be written as follows: \bigskip

\begin{eqnarray} \label{eq:3}
	M(x) = 
	\begin{dcases}
		0,                                            & \text{for } 0 \leq x < 1 \\\\
		1,                                            & \text{for } x = 1 \\\\
		1 + \frac{2}{x - 1} \int_{0}^{x - 1} M(t) dt, & \text{for } x > 1
	\end{dcases}
\end{eqnarray}\medskip

which has the form of a recurrence. 






\section{Applications}

The main application of the parking problem is to the theory of random 
sequential adsorption (RSA), which models the adsorption of particles onto 
a solid substrate. It is important to make the distinction between 
\emph{adsorption} and \emph{absorption} - adsorption involves the surface 
of the material involved, whereas absorption involves the full volume of 
the material. \bigskip

Some examples of RSA are adsorption of gas molecules, polymer chains, 
latex particles, bacteria, proteins, and colloidal particles (insoluble 
particles contained within a suspension) onto a surface. Another important 
application is when applied to genome sequencing where a newly-arriving 
sequenced clone is allowed to partially overlap an existing sequenced 
clone - in which case the parking problem with overlap the most relevant 
model. In some cases adsorbed particles can also be released from a 
surface, a process know as \emph{desorption}, and in this case the 
reversible parking problem is the most relevant model. \bigskip

In order to maximize the adsorption of particles, a clear understanding of 
the problem is required, and to that end we need to be able to faithfully 
model the process. This paper will offer a survey of the approaches to 
modelling the one-dimensional parking problem, and provide a different 
approach to both validate the approaches used to model the process, and to 
calculate the associated constants. \bigskip











