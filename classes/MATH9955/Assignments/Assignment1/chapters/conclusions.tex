
\chapter{Conclusions}











\section{Remarks}

The kinetic approach is wholly satisfying, and leaves the way open to further research. 
This author was struck by the elegance of the derivations, and spent some time considering 
whether or not there was some underlying conservation law other than the obvious: \bigskip

\[
	\int_{0}^{\infty} x P(x, t) dx + \int_{0}^{\infty} P(x, t) dx = 1
\]\medskip

i.e that the normalized sum of the car lengths and gap lengths equals 1. And with respect to 
the generalizations of the kinetic framing of the problem, once again we see two beautifully 
elegant solutions. In particular, the reversible problem, and it's condition for equilibrium: \bigskip

\[
	\frac{\partial P(x, t)}{\partial t} = 0
\]\medskip

which would lead one to believe that there are hidden symmetries. \bigskip

We have demonstrated that the general approach of validating theory through simulations has 
been useful. When it comes to modelling a problem that does not lend itself easily to 
empirical verification, such as the parking problem, a simulation can tell you if your theory 
makes sense very quickly. \bigskip

Also, we have shown that simulations can help to provide statistical properties relating to 
the associated constants, which in turn can help establish whether a process is being 
inhibited by some unknown factor - for example, if an adsorption yield in some industrial 
process falls outside of $2$ standard deviations of the mean yield from a simulation based 
on the existing conditions, then some investigation might be warranted. \bigskip











\section{Future Research}

In this author's view there is scope for further research into the derivation of the 
time-independent limit, in the one-dimensional case, which does not take R\'enyi's 
approach. Also, the jamming limit $C_R$ feels like something universal, akin to $e$ 
or $\phi$, or even $\pi$. It feels like the surface was only scratched and a much more 
satisfying number-theoretic approach could be taken. This author would welcome such 
research. \bigskip

This author also believes that there may be more interesting results, and possibly 
hidden symmetries found, from studying this problem in the context of distributions 
- both in the specific probabilistic, and more general special function, sense. \bigskip

Higher dimensional problems, in two and three dimensions, would be a logical next 
research topic - investigating what (if any) relationship exists between two and 
three dimensional packing problems and $C_R$. \bigskip

Sadly, one generalization I did not have time to look at in much depth is 
\emph{Competitive RSA of a binary mixture}, where cars of two different lengths 
compete for parking spaces. This is a more complicated problem mathematically, and 
is actually more difficult to simulate, but might have more general, or practical, 
applications. This author feels this subject is certainly worthy of more attention. \bigskip










