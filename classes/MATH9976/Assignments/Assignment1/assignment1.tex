\documentclass{report}

\usepackage{amsmath}
\usepackage{amssymb}
\usepackage{mathtools}
\usepackage{titlepic}
\usepackage{url}
\usepackage{natbib}
\usepackage{graphicx}
\usepackage{listings}
\usepackage{color}
\usepackage{lmodern}
\usepackage[T1]{fontenc}

\setlength\parindent{0pt}
\graphicspath{{./images/}}

\renewcommand{\baselinestretch}{1.5}







\begin{document}

\titlepic{\includegraphics[scale=0.5]{DIT_logocol}}
\title{The Weighted Random Graph Model}
\author{Jerry Kiely\\
	\\
	School of Mathematical Sciences\\
	Dublin Institute of Technology\\
	Dublin 8\\
	Ireland\\
	\\
	\texttt{d16126734@mydit.ie}}
\date{\today}
\maketitle


\tableofcontents

\newpage










\begin{abstract}
The author introduced the weighted random graph (WRG) model, which represents "the weighted counter-part 
of the Erdos-R\'{e}nyi random graph model", and which provides "fundamental insights into more complicated 
weighted networks". 
\end{abstract}











\chapter{Background}

A graph $G(V, E)$ may be considered a collection of vertices and edges. The vertices, or the nodes, are 
the things we are modelling - computers, people, towns, etc. The edges are the relationships between the 
things we are modelling - network cables, relationships, roads, etc. \bigskip

\begin{figure}[h]
	\centering
	\includegraphics[scale = 0.5]{random}
	\caption{A Random Graph}
	\label{fig:random}
\end{figure}\medskip

A graph can be \textit{directed} or \textit{undirected}. For example, in an undirected graph representing 
a network, two nodes $a$ and $b$ (representing computers) connected by an edge (a network cable) can be 
represented by an unordered pair $\{a, b\}$ - unordered because the connection goes in both directions. In 
a directed graph representing the world wide web, two nodes $a$ and $b$ (representing pages) connected by 
and edge (a hyperlink) can be represented by an ordered pair $(a, b)$ - ordered because the connection 
goes in one direction. \bigskip

Also a graph can be weighted or unweighted - where edges between vertices can have an associated weight. 
For example, in a graph representing towns in a country, two nodes $a$ and $b$ representing towns would 
have an edge $\{a, b\}$ with an associated weight representing the distance between the two towns. \bigskip














\chapter{Methods}


When we talk about \textit{Random Graphs} (\cite{book1, article1}) we refer to probability distributions on 
graphs. These graphs can be directed, undirected, weighted, unweighted, etc. Generating ramdom graphs can 
be done in a number of different ways. For example: \bigskip

\begin{itemize}
	\item fix the number of nodes $n$ and the number of edges $m$
	\item number the nodes $1$ to $n$
	\item select two nodes at random forming an edge
	\item continue $m$ times 
\end{itemize}\medskip

another approach: \bigskip

\begin{itemize}
	\item fix the number of nodes $n$ and a probability threshold $p$
	\item number the nodes $1$ to $n$
	\item for each pair of nodes generate a random number
	\item form an edge if the random number is greater than $p$ 
\end{itemize}\medskip

There are many statistics of interest around random graphs: \bigskip

\begin{itemize}
	\item the degree distribution of the vertices (popularity, centre)
	\item the number of edges (relationships)
	\item the number of sub-graphs (triangles, etc.)
	\item the number of connected components
\end{itemize}\medskip















\chapter{Results}

Draw an edge between two nodes with weight $w$ with probability: \bigskip

\[
	q_{ij}(w) = (y_1 y_j)^w (1 - y_1 y_j)
\]\medskip

making a few assumptions we can generalise this as: \bigskip

\[
	q(w) = p^w (1 - p)
\]\medskip

it is easy to show that the likelihood maximising choice is: \bigskip

\[
	p^* = \frac{2W}{N(N - 1) + 2W}
\]\medskip

the expected weight of any edge (\cite{article2}) is: \bigskip

\[
	\langle w \rangle = \sum_{w = 0}^{+\infty} q(w) w = \frac{p}{1 - p}
\]\medskip

the variance is: \bigskip

\[
	\langle w^2 \rangle - \langle w \rangle^2 = \frac{p}{(1 - p)^2}
\]\medskip

the degree distribution of the whole network is: \bigskip

\[
	P(k) = \binom{N - 1}{k} p^k (1 - p)^{N - 1 - k}
\]\medskip

the strength distribution is: \bigskip

\[
	P(s) = \binom{N - 2 + s}{N - 2} p^s (1 - p)^{N - 1}
\]\medskip

If all edges with weight smaller than $w$ are removed, it is clear that the remaining edges form an 
unweighted projection equivalent to an ER random graph with probability: \bigskip

\[
	p_{w}^{+} \equiv \sum_{v = w}^{+\infty} q(v) = p^w
\]\medskip

similarly, if all edges greater than $w$ are removed, we have: \bigskip

\[
	p_{w}^{-} \equiv \sum_{v = 1}^{w} q(v) = p - p^{w + 1}
\]\medskip

and so on. The Critical weight: \bigskip

\[
	w_{c}^{+} \equiv - \frac{\ln{N}}{\ln{p}}
\]\medskip

removing edges of weight greater than or equal to $w_{c}^{+}$ the network will fragment. \bigskip

\[
	\lim_{N \to \infty} w_{c}^{+} = \frac{1}{\alpha} \qquad \text{ if } \qquad p \sim N^{- \alpha}
\]\medskip

for strong link removal the critical weight is: \bigskip

\[
	w_{c}^{+} \equiv - \frac{\ln{(p - 1/N)}}{\ln{p}} - 1 \simeq 0
\]\medskip











\chapter{Comments}

The study of random graphs has many applications. For example, the study of the passage of infectious 
diseases can be helped by the study of random graphs - modelling with a random graph may be the only 
possibility when the actual graph is not known in advance, or where the actual graph may change 
regularly. \bigskip

Also, when studying a graph, by comparing it to known properties of a random graph we can easily decide 
if an observed property is worth studying - in other words, if an observed property of a graph is present 
within a random graph, then it is likely not anomalous, and does not warrant further research. \bigskip

Weighted random graphs provide a more complete description of graphs used to model many different 
phenomena, and arise naturally when describing networks that model transport for example. \bigskip


\newpage








\bibliographystyle{plain}
\bibliography{bibtex/bibliography}

\end{document}
