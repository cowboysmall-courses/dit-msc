%%%%%%%%%%%%%%%%%%%%%%%%%%%%%%%%%%%%%%%%%%%%%%%%%%%%%%%%%%%%%
%%%%% Sample examination layout for 		       	%%%%%
%%%%%          dit_maths_exam.sty 			%%%%%
%%%%%							%%%%%
%%%%%	V3 September 2015    				%%%%%
%%%%%	- Use new sty file to mirror CoSH template 	%%%%%
%%%%%   - include bw option for black & white logo      %%%%%
%%%%%	- instructions on new programmes		%%%%%
%%%%%							%%%%%
%%%%%	V2.1 October 2014    				%%%%%
%%%%%	- Fix bug using old sty file naming convention 	%%%%%
%%%%%							%%%%%
%%%%%	V2 October 2014    				%%%%%
%%%%%	- Remove reference to Springboard programmes 	%%%%%
%%%%%	- included new external examiners		%%%%%
%%%%%							%%%%%
%%%%%	V1.1 October 2013    				%%%%%
%%%%%	- New external examiner instruction included	%%%%%
%%%%%	- Two new rubrics: DT205/DT220/1 Programming	%%%%%
%%%%%							%%%%%
%%%%%	V1 February 2013    				%%%%%
%%%%%	- Original version				%%%%%
%%%%%%%%%%%%%%%%%%%%%%%%%%%%%%%%%%%%%%%%%%%%%%%%%%%%%%%%%%%%%

\documentclass[a4paper,12pt]{article}

	%\usepackage{amsmath,color,graphicx,epstopdf,wrapfig,enumerate}
	\usepackage{amsmath,enumerate,graphics,graphicx,color,amsthm}
	%\usepackage{venturis2}
	%\usepackage[T1]{fontenc}
	\usepackage[adobe-utopia]{mathdesign}
	\usepackage[T1]{fontenc}
	\usepackage{enumitem}
	\newcommand{\lqs}{\leqslant}
	\newcommand{\gqs}{\geqslant}
	\newcommand{\R}{\mathbb{R}}
	\definecolor{gray}{RGB}{128,128,128}
	\definecolor{lime}{RGB}{153,204,0}
	\newtheoremstyle{Qstyle}{}{20pt}{}{}{\bfseries}{:\quad}{ }{}
	\theoremstyle{Qstyle}
	\newtheorem{q}{Q\hspace{-2pt}}
%%%%%%%%%%%%%%%%%%%%%%%%%%%%%%%%%%%%%%%%%%%%%%%%%%%%%%%%%%%%%
%%%%% 	ENSURE LATEST VERSION OF STYLE IS USED HERE	%%%%%
%%%%%   ---use option argument bw to use b&w logo       %%%%%
%%%%%%%%%%%%%%%%%%%%%%%%%%%%%%%%%%%%%%%%%%%%%%%%%%%%%%%%%%%%%

%\usepackage{dit_maths_exam_V3}		%use for color logo
\usepackage[bw]{dit_maths_exam_V3}	%use for bw logo

%%%%%%%%%%%%%%%%%%%%%%%%%%%%%%%%%%%%%%%%%%%%%%%%%%%%%%%%%%%%%
%%%%% include optional packages here 		       	%%%%%
%%%%%%%%%%%%%%%%%%%%%%%%%%%%%%%%%%%%%%%%%%%%%%%%%%%%%%%%%%%%%
\usepackage{}

%%%%%%%%%%%%%%%%%%%%%%%%%%%%%%%%%%%%%%%%%%%%%%%%%%%%%%%%%%%%%
%%%%% include local command definitions here	       	%%%%%
%%%%%%%%%%%%%%%%%%%%%%%%%%%%%%%%%%%%%%%%%%%%%%%%%%%%%%%%%%%%%


%%%%%%%%%%%%%%%%%%%%%%%%%%%%%%%%%%%%%%%%%%%%%%%%%%%%%%%%%%%%%
%%%%% enter examination code from timetable	   	%%%%%
%%%%% ---ensure all concurrent exams are included---	%%%%% 
%%%%%%%%%%%%%%%%%%%%%%%%%%%%%%%%%%%%%%%%%%%%%%%%%%%%%%%%%%%%%
\renewcommand{\code}{W260/322}


%%%%%%%%%%%%%%%%%%%%%%%%%%%%%%%%%%%%%%%%%%%%%%%%%%%%%%%%%%%%%
%%%%% enter module title including module code 	       	%%%%%
%%%%% ---format: MATH XXXX: Title of Module		%%%%%
%%%%%%%%%%%%%%%%%%%%%%%%%%%%%%%%%%%%%%%%%%%%%%%%%%%%%%%%%%%%%
\renewcommand{\mtitle}{MATH9973: Numerical Methods for Differential Equations}

%%%%%%%%%%%%%%%%%%%%%%%%%%%%%%%%%%%%%%%%%%%%%%%%%%%%%%%%%%%%%
%%%%% enter all programmes 				%%%%%
%%%%% ---separate with \\[10pt] & use the definitions:	%%%%%
%%%%%		\dt{205} \dt{220} 			%%%%%
%%%%%		\dt{6248} \dt{7248} \dt{8248} 		%%%%%
%%%%%		\dt{8998}				%%%%%
%%%%%		\dt{9205} (includes DT9206)		%%%%%
%%%%%		\dt{9209} (includes DT9210)		%%%%%
%%%%%		\dt{9211} (includes DT9212)		%%%%%
%%%%%[only for legacy/repeat students 2015/2016] 	%%%%%
%%%%%		\dt{234} \dt{238} 			%%%%%
%%%%%%%%%%%%%%%%%%%%%%%%%%%%%%%%%%%%%%%%%%%%%%%%%%%%%%%%%%%%%
\renewcommand{\progs}{\dt{9205} \\[10pt] } 

%%%%%%%%%%%%%%%%%%%%%%%%%%%%%%%%%%%%%%%%%%%%%%%%%%%%%%%%%%%%%
%%%%% enter stage number		 	       	%%%%%
%%%%%  	- use second definition if no stage  		%%%%%
%%%%%%%%%%%%%%%%%%%%%%%%%%%%%%%%%%%%%%%%%%%%%%%%%%%%%%%%%%%%%
%\renewcommand{\stage}{Stage 1} 	%comment if no stage
\renewcommand{\stage}{\vspace{-0.25in}}	%comment if stage exists

%%%%%%%%%%%%%%%%%%%%%%%%%%%%%%%%%%%%%%%%%%%%%%%%%%%%%%%%%%%%%
%%%%% enter examination session	& ACADEMIC year		%%%%%
%%%%% ---use following definitions:			%%%%%
%%%%%		\winter \summer \autumn			%%%%%
%%%%% ---use format YYYY/YYYY				%%%%%
%%%%%%%%%%%%%%%%%%%%%%%%%%%%%%%%%%%%%%%%%%%%%%%%%%%%%%%%%%%%%
\renewcommand{\session}{\autumn}
\renewcommand{\acyear}{2015/2016}

%%%%%%%%%%%%%%%%%%%%%%%%%%%%%%%%%%%%%%%%%%%%%%%%%%%%%%%%%%%%%
%%%%% enter examiners					%%%%%
%%%%% ---use following definitions:			%%%%%
%%%%%		\appleby \murphy \riordan \tuite	%%%%%
%%%%% Head of School will be included automatically	%%%%%
%%%%%%%%%%%%%%%%%%%%%%%%%%%%%%%%%%%%%%%%%%%%%%%%%%%%%%%%%%%%%
\renewcommand{\examiner}{Dr. John S. Butler}
\renewcommand{\external}{\riordan}  	%comment if no external
%\renewcommand{\external}{ } 		%comment if external exists

%%%%%%%%%%%%%%%%%%%%%%%%%%%%%%%%%%%%%%%%%%%%%%%%%%%%%%%%%%%%%
%%%%% enter date time					%%%%%
%%%%% ---use following formats:				%%%%%
%%%%%	Thursday, 22 August 2014	2.00 -- 4.00 pm	%%%%%
%%%%%%%%%%%%%%%%%%%%%%%%%%%%%%%%%%%%%%%%%%%%%%%%%%%%%%%%%%%%%
\renewcommand{\exdate}{}
\renewcommand{\extime}{}
%\newcommand{\duration}{2 hours}
%\newcommand{\duration}{3 hours}
\newcommand{\duration}{2 hour}
	\newcommand*{\Comb}[2]{{}^{#1}C_{#2}}%
	\newcommand*{\Pick}[2]{{}^{#1}P_{#2}}%
%%%%%%%%%%%%%%%%%%%%%%%%%%%%%%%%%%%%%%%%%%%%%%%%%%%%%%%%%%%%%
%%%%% rubrics: comment/uncomment as necessary		%%%%%
%%%%% ---there are two rubric lines rubrica & rubricb	%%%%%
%%%%% ---use standard wordings given			%%%%%
%%%%%%%%%%%%%%%%%%%%%%%%%%%%%%%%%%%%%%%%%%%%%%%%%%%%%%%%%%%%%
%\renewcommand{\rubrica}{Attempt three questions only}
\renewcommand{\rubrica}{Full marks may be obtained by answering three questions.  Candidate's three best questions will contribute to their final mark.}
%\renewcommand{\rubrica}{Attempt all qustions}
%\renewcommand{\rubrica}{Attempt question 1 and any two other questions}
%\renewcommand{\rubrica}{Answer four questions}
%\renewcommand{\rubrica}{Attempt question 1 and any three other questions}
\renewcommand{\rubricb}{All questions carry equal marks}
%\renewcommand{\rubricb}{Answer at least two questions from each section.  All questions carrry equal marks}
%\renewcommand{\rubricb}{Question 1 carries 40 marks.  All other questions carry 20 marks each.}

%%%%%%%%%%%%%%%%%%%%%%%%%%%%%%%%%%%%%%%%%%%%%%%%%%%%%%%%%%%%%
%%%%% comment out those that are not permitted		%%%%%
%%%%% PLEASE ONLY INCLUDE ITEMS THAT ARE NECESSARY 	%%%%%
%%%%% AND APPROPRIATE					%%%%%
%%%%%%%%%%%%%%%%%%%%%%%%%%%%%%%%%%%%%%%%%%%%%%%%%%%%%%%%%%%%%
\calculators	%comment out if not permitted
\tables		%comment out if Mathematical tables not to be provided
%\stats		%comment out if New Cambridge Statistical Tables not to be provided


%%%%%%%%%%%%%%%%%%%%%%%%%%%%%%%%%%%%%%%%%%%%%%%%%%%%%%%%%%%%%%%%%%%%%%%%%%
\begin{document}
%
\makecover

%%%%%%%%%%%%%%%%%%%%%%%%%%%%%%%%%%%%%%%%%%%%%%%%%%%%%%%%%%%%%%%%%%%%%%%%%%
%%%%%%%%%%%%%%%%%%%%%%%%%%%%%%%%%%%%%%%%%%%%%%%%%%%%%%%%%%%%%%%%%%%%%%%%%%



%%%%%%%%%%%%%%%%%%%%%%%%%%%%%%%%%%%%%%%%%%%%%%%%%%%%%%%%%%%%%
%%%%% place questions in the question environment	%%%%%
%%%%% 	\begin{question}{total marks} ...\end{question}	%%%%%
%%%%% use \begin{enumerate} for sub(sub)sections	%%%%%
%%%%% use \mrk{**} for mark allocations within question	%%%%%
%%%%%%%%%%%%%%%%%%%%%%%%%%%%%%%%?%%%%%%%%%%%%%%%%%%%%%%%%%%%%%
\begin{enumerate}

\begin{question}{33}

\begin{enumerate}
	\item
	Derive the Euler approximation show it has a local truncation error of $O(h)$ of the Ordinary Differential Equation
	\[y^{'}(x)=f(x,y) \]
	with initial condition
	\[y(a)=\alpha. \]
	\mrk{7}
	\item 
	Suppose $f$ is a continuous and satisfies a Lipschitz condition with constant
	L on $D=\{(t,y)|a\leq t \leq b, -\infty < y < \infty \}$ and that a constant M
	exists with the property that 
	\[ |y^{''}(t)|\leq M. \]
	Let $y(t)$ denote the unique solution of the IVP
	\[ y^{'}=f(t,y) \ \ \ a\leq t \leq b \ \ \ y(a)=\alpha \]
	and $w_0,w_1,...,w_N$ be the approx generated by the Euler method for some
	positive integer N.  Then show for $i=0,1,...,N$
	\[ |y(t_i)-w_i| \leq \frac{Mh}{2L}|e^{L(t_i-a)}-1| \]
	You may assume the two lemmas:\\
	If s and t are positive real numbers $\{a_i\}_{i=0}^{N}$ is a sequence satisfying $a_0 \geq \frac{-t}{s}$ and $a_{i+1} \leq (1+s)a_i +t $
	then
	\[a_{i+1} \leq e^{(i+1)s}\left(a_0+\frac{t}{s}\right)-\frac{t}{s} \]
	
	For all $ x \geq 0.1$ and any positive m we have \[0\leq (1+x)^m \leq e^{mx}\]
	\mrk{14}
	\item
	Use the shooting method combined with the Euler's method to estimate the solution of
	\[ y^{''}-y=2x, \ \ \ y(0)=-2 y(1)=-0.9 \]
	at x=1, using $h=0.25$.
	\mrk{12}
\end{enumerate}

\end{question}

\begin{question}{33}
	\begin{enumerate}
		\item
		Derive the Adams Bashforth two step method and its truncation error which is of the form
		\[w_0=\alpha_0 \ \ \ w_1=\alpha_1 \]
		\[w_{i+1}=w_i + \frac{h}{2}[3f(t_i,w_i)-f(t_{i-1},w_{i=1})] \]
		\mrk{10}
		\item
		Define the terms consistent and convergent methods for a multistep method.
		\mrk{5}
		\item
		Define the terms strongly stable, weakly stable and unstable with respect to the
		characteristic equation.
		\mrk{6}
		\item
		Show that the Adams Bashforth two step method is  stongly stable.
		\mrk{6}
		\item
		Use an Adams-Basforth method of your choice to approximate the solution to the initial value problem.
		\[y^{'}=1+(t-y)^2, \ \  2\leq t \leq 3,  \ \ \ y(2) = 1 \]
		with $h=0.2$.
		\mrk{7}
		\end{enumerate}
\end{question}

\begin{question}{33}
	\begin{enumerate}
		\item
		Approximate the Poisson equation 
		\[ -\nabla^2U(x,y)=f(x,y) \ \ \ \ \ \ (x,y) \in \Omega=(0,1)\times (0,1) \]
		with boundary conditions
		\[U(x,y) = g(x,y) \ \ \ \ \ \ \ \  (x,y)\in\delta\Omega-boundary \]
		using the five point method.  Sketch how the finite difference scheme may be 
		rewritten in the form $Ax=b$, where A is a sparse
		$N^2\times N^2$ matrix, $b$ is an $N^2$ component matrix and $x$ is an $N^2$
		component vector of unknowns.
		(Assume your 2d discretised grid contains $N$ components in the $x$ and $y$ direction).
		%with boundary conditions
		%\[ u(0,t)=u(\pi,t)=0, \ \ 0<t \]
		%and initial conditions
		\mrk{8}
		%\item
		%Prove that the five-point difference analog $-\nabla^2_h$ is consistent to order 2 with $-\nabla^2$.
		
		\item Prove (DISCRETE MAXIMUM PRINCIPLE).
		if $\nabla^2_hV_{ij}\geq 0$ for all points $(x_i,y_j) \in \Omega_h$, then
		\[ \max_{(x_i,y_j)\in\Omega_h}V_{ij}\leq  \max_{(x_i,y_j)\in\partial\Omega_h}V_{ij}\]
		If $\nabla^2_hV_{ij}\leq 0$ for all points $(x_i,y_j) \in \Omega_h$, then
		\[ \min_{(x_i,y_j)\in\Omega_h}V_{ij}\geq  \min_{(x_i,y_j)\in\partial\Omega_h}V_{ij}\]
\mrk{14}
		\item
		Hence prove:\\
		Let $U$ be a solution to the Poisson equation and let $w$ be the grid function
		that satisfies the discrete analog
		\[-\nabla_h^2w_{ij}=f_{ij} \ \ \mbox{ for } (x_i,y_j)\in\Omega_h, \]
		\[w_{ij}=g_{ij} \ \ \mbox{ for } (x_i,y_j)\in\partial\Omega_h. \]
		Then there exists a positive constant $K$ such that
		\[||U-w||_{\Omega}\leq KMh^2 \]
		where
		\[ M=\left\{
		\left|\left|\frac{\partial^4 U}{\partial x^4} \right|\right|_{\infty},
		\left|\left|\frac{\partial^4 U}{\partial x^3\partial y} \right|\right|_{\infty},
		,...,
		\left|\left|\frac{\partial^4 U}{\partial y^4} \right|\right|_{\infty}
		\right\}
		\]
		You may assume:\\
		\textbf{Lemma}\\
		If the grid function $V:\Omega_h\bigcup\partial\Omega_h\rightarrow R$ satisfies the boundary condition $V_{ij}=0$ for $(x_i,y_j)\in \partial\Omega_h$, then
		\[||V||_{\Omega}\leq \frac{1}{8}||\nabla_h^2V||_{\Omega} \]
	\mrk{11}
		\end{enumerate}
	
\end{question}

\begin{question}{33}
	
	Consider the second order differential equation 
	\[\frac{d^2u}{dx^2}+u=x \ \]
	with boundary conditions
	\[u(0)=0 \ \ \ \ \ u(1)=0 \]
	\begin{enumerate}
		\item
		Show that the solution $u(x)$ of this equation satisfies the weak form
		\[ \int_{0}^{1} dx \left(-\frac{du}{dx}\frac{dv}{dx}+uv-xv \right) = 0\]
		for all $v(x)$ which are sufficiently smooth and which satisfy
		\[v(0)=0 \ \ \ \ \ v(1)=0 \]
		\mrk{13}
		\item
		By splitting the interval $x\in [0,1]$ into $N$ equal elements of size $h$, where
		$Nh=1$, one can define nodes $x_i$ and finite element shape functions as 
		follows
		\[x_i=ih \]
		\[\phi_i(x)=\left\{\begin{array}{ll} 
		0 & 0\leq x \leq x_{i-1}\\
		\frac{x-x_{i-1}}{h} & x_{i-1}\leq x \leq x_{i}\\
		\frac{x_{i+1}-x}{h} & x_{i}\leq x \leq x_{i+1}\\
		0 & x_{i+1}\leq x \leq 1\\
		\end{array} \right. \]
		A finite element approximation to the differential equation is obtained by approximating $u(x)$ and $v(x)$ with linear combinations of these finite element shape
		functions, $\phi_i$, where
		\[ u_{n} = \sum_{i=1}^{N-1}\alpha_i \phi_i(x) \]
		\[ v_{n} = \sum_{j=1}^{N-1}\beta_j \phi_j(x) \]
		Show that the equation which results from this approximation has the form
		\[K\alpha= F \]
		where K is an $N-1 \times N-1$ sparse matrix, $F$ is an $N-1$ component vector
		and $\alpha$ is an $N-1$ component vector of unknown co-efficient $\alpha_i$.
		\mrk{14}
		\item
		What structure does the matrix $K$ have?
		Evaluate the first component of the main diagonal of $K$.
		\mrk{8}
\end{enumerate}
	
\end{question}


\end{enumerate}

\end{document}